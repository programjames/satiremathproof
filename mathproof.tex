\documentclass{article}
\usepackage{amsmath, amssymb}
\usepackage{titling}
\usepackage{xcolor, soul}
\author{James Camacho}
\title{On the Irrationality of Roots of 2}
\date{December 2, 1992}
\begin{document}
	\maketitle
	\par
	It is well known that $\sqrt2$ is irrational. Here we prove that $\sqrt[n]{2}$ is also irrational for integers $n > 2$ using induction on $n$ with the base case $n=3$. First we need to prove a few preliminary lemmas.
	\newline
	\hrule
	\vspace{5mm}
	\noindent
	\textbf{Lemma 1:} $\sqrt[3]{2}$ is irrational.
	\par
	\emph{Proof:} Suppose for the sake of contradiction that $\sqrt[3]{2}$ is rational. So it can be expressed as $\frac{p}{q}$ where $p\ge q\ge 1 \in\mathbb{N}$ and $(p, q) = 1.$ So,
	$$2 = \frac{p^3}{q^3}$$
	so
	\begin{equation}
	p^3 = q^3 + q^3.
	\end{equation}
	There are no solutions to the equation $a^3 = b^3 + c^3$ for positive integers $a, b,$ and $c$ (confirm; the proof should not take more than the margins of this paper), so there are no solutions to $(1)$. $\square$
	\vspace{5mm}
	\newline
	\textbf{Lemma 2:} If $r$ satisfies $\frac{1}{r} = \sqrt[n]{2r}$ then $r$ is irrational.
	\par
	\emph{Proof:} This implies that $r$ is a root to
	$$\frac{1}{x^n} = 2x\Longleftrightarrow 2x^{n+1} - 1 = 0.$$
	By the Rational Root Theorem the only possible rational roots are $\pm 1$  and $\pm \frac{1}{2}$ but none of them are roots to this particular equation (confirm). Therefore $r$ must be irrational. $\square$
	\vspace{5mm}
	\newline
	\textbf{Lemma 3:} If $\sqrt[n]{2}$ is irrational, so is $\sqrt[n+1]{2}$.
	\par
	\emph{Proof:} Suppose for the sake of contradiction that $\sqrt[n]{2}$ is irrational but $\sqrt[n+1]{2}$ is not. So $\sqrt[n+1]{2} = \frac{p}{q}$ for some positive integers $p\ge q$ and $(p, q) = 1$. So,
	$$\sqrt[n]{2} = \left(\sqrt[n+1]{2}\right)^{1+\frac{1}{n}} = \frac{p}{q}\cdot \sqrt[n]{\frac{p}{q}}$$
	So
	$$\frac{p}{q} = \sqrt[n]{\frac{2q}{p}}$$
	is rational. But by putting $r=\frac{q}{p}$, we get from Lemma 2 that there is no rational solution to that equation! There is a contradiction, so $\sqrt[n+1]{2}$ is also irrational. $\square$
	\newline
	\hrule
	\vspace{5mm}
	\par
	Now we can complete the proof. Lemma 1 shows that the base case, $n=3$, holds. Lemma 3 shows that the inductive step holds. So, the claim is true. $\sqrt[n]{2}\in\mathbb{P}\ \ \forall n\in\mathbb{Z}_{\ge 3}.\qquad \blacksquare$
	
\end{document}